\section*{Снова потоки}
\begin{enumerate}
	\item Разберитесь с пересчётом потенциалов при поиске потока.
	
	\item Дан массив, найдите $k$ непересекающихся возрастающих последовательностей максимальной длины за $O(kV^2)$.
	
	\item Дан граф, на каждом ребре написано $2$ числа $L$ и $R$ и $c$. По каждому ребру может течь не более
	чем $R$, но не менее, чем $L$ жидкости. Найдите:
	\begin{enumerate}
		\item произвольную циркуляцию
		\item произвольный поток
		\item максимальный поток
		\item поток минимальной стоимости.
	\end{enumerate}
	
	\item Есть k одинаковых автоматов и n заданий. Про каждое задание известно, во сколько его нужно начать делать, во сколько закончить, а также его стоимость. Каждый автомат может выполнять только одно задание в каждый момент времени. Нужно выполнить задания максимальной суммарной стоимости. $O(kn \log n)$.
\end{enumerate}
