\section*{Суффиксные структуры}
\begin{enumerate}
	\item Найдите усреднённое по всем парам суффиксов значение $lcp$ для строки длины $n$ с помощью	суффиксного массива за $O(n)$.
	\item[3.] Найдите максимальный рефрен — такую подстроку строки $s$, что количество ее вхождений (возможно пересекающихся) помноженных на ее длину — максимально. Решите двумя способами: суффиксными деревом и массивом.
	
	\textbf{Решение.} 
	
	\begin{itemize}
		\item \textit{Суффиксное дерево:} построим суффиксное дерево. Для каждой вершины вычислим две величины: длина 
		строки, соответствующей пути от корня до вершины - $length[v]$, и количество листьев в поддереве вершины 
		$leaves[v]$. Первая величина соответствует длине, вторая - количество вхождений, $v$ - некоторая подстрока. 
		
		Таким образом, ответом на задачу будет являться такая $v$, для которой значение $length[v] \cdot leaves[v] 
		\to \max$.
		
		Сложность решения составляется $O(n)$, т.к. нужно построить дерево и выполнить обход суффиксного дерева.
		
		\item \textit{Суффиксный массив:} Построим суффиксный массив. Осталось найти максимум среди отрезков $[l, r]$, где $1 \leq l \leq r \leq n$:
		\begin{equation*}
			(r - l + 1) \cdot \min \limits_{i \in [l, r)} \left[ lcp(p_i, p_{i + 1}) \right]
		\end{equation*}
		
		Так можно сделать, т.к. префикс суффикса $p_l$ длины $\min \limits_{i \in [l, r)} \left[ lcp(p_i, p_{i + 1}) \right]$ входит в строку по крайней мере $(r - l + 1)$ раз. Т.к. сложность не требуется, то можно воспользоваться произвольным алгоритмом поиска $lcp$, и вычислять заданное соотношение "в лоб".
		
	\end{itemize}
	\item[4.] Дан набор строк $s_i$ суммарной длины $n$. Для каждой $s_i$ найдите $min$ по длине подстроку, которая не встречается в других. $O(n)$.
	\item[5.] Даны $k$ строк суммарной длины $n$. Найдите $p$-ю лексикографически общую их подстроку за $O(n)$.
\end{enumerate}
