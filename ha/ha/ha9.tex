\section*{Потоки и разрезы}
\begin{enumerate}
	\item Дан граф и выделенные вершины $s, t$. Нужно проверить, правда ли существует единственный минимальный s-t разрез.
	\begin{enumerate}
		\item $O(poly(V, E))$
		
		\textbf{Решение.} Заметим, что минимальный $s-t$ разрез мы можем найти при помощи алгоритма 
		Форда-Фалкерсона, указав пропускную способность каждого ребра равной единице (Обозначим размер минимального 
		разреза $F$). Если таких минимальных разрезов несколько, то уменьшив до 0 пропускную способность какого-то 
		из рёбер мин. разреза, мы получим другой разрез с потоком $F$. Осталось понять, какому ребру нужно 
		уменьшать пропускную способность (нужно выбрать то, которое входит в один мин. разрез, и не входит в 
		другой). Раз уж мы не очень ограничены в сложности, то переберём все ребра исходного мин. разреза(их не 
		более $E$), уменьшим пропускную способность каждого из рёбер, заново найдём поток, если он равен $F$, то 
		разрез не единственный, если после уменьшения пропускной способности всех рёбер поток уменьшался, то 
		минимальный $s-t$ разрез в исходном графе единственен. Сложность описанного подхода, очевидно, $O(poly(V, 
		E))$
		\item $O(E)$ при условии, что нам уже известен максимальный поток (с доказательством).
	\end{enumerate}
	
	\item В неориентированном графе без кратных рёбер необходимо удалить минимальное число рёбер так, чтобы 
	увеличилось количество компонент связности. $O(V \cdot Flow)$. Оцените время работы	того же алгоритма более 
	точно как $O(E^2)$.
	
	% TODO: описать способ ориентации ребер графа.
	
	\textbf{Решение.} Сразу скажем, что для решения задачи достаточно научиться искать мин. количество рёбер, 
	которое нужно удалить чтобы развалить граф для связных графов, т.к. для несвязных достаточно предпосчитать 
	компоненты связности за $O(DFS) = O(V + E)$, а потом искать минимальные разрезы для каждой компоненты. 
	Сложность при этом не изменится.
	
	Будем искать минимальные разрез следующим образом. Рассмотрим некоторый минимальный разрез для исходного 
	графа. Очевидно, что он разобьёт множество вершин графа на две непустые части: $S, V / S$. Теперь заметим, 
	что для того, чтобы найти этот разрез достаточно выбрать в качестве вершины истока некоторую вершину $s \in 
	S$, а в качестве вершины - стока $t \in V / S$, после чего лишь останется всем рёбрам дать пропускную 
	способность, равную единице, и найти максимальный поток. Осталось понять, как выбрать вершины $s, t$. 
	Очевидно, это можно сделать так: $s$ выберем произвольно, $t$ переберём все остальные. Хотя бы одно значения 
	$t$ окажется в другой части разбиения множества вершин графа, и именно в этом случае мы и получим минимальный 
	разрез. То есть нужно запустить $V - 1$ алгоритм поиска максимального потока. А заметив, что ребра графа 
	имеют пропускные способности равные единице, можем получаем возможность искать максимальный поток за $O(VE)$. 
	Т.к. кратных рёбер нет, то $V^2 = O(E)$, значит итоговую сложность можно оценить как $O(V^2E) = O(E^2)$. 
	
	\item Есть ориентированный граф с начальной и конечной вершинами. В начальной вершине есть $K$ грузовиков. 
	Грузовикам нужно попасть в конечную вершину. Время дискретно. За единицу времени каждый грузовик или стоит на 
	месте, или перемещается в одну из соседних вершин. В любой вершине могут одновременно стоять несколько 
	грузовиков. По любому из рёбер в каждый момент времени должен ехать не более чем один грузовик. Минимизируйте 
	время, когда все грузовики окажутся в конечной вершине.
	\begin{enumerate}
		\item $O(poly(V, E, K))$
		\item $O(K(V + K)E)$
	\end{enumerate}
	
	\item Есть $n$ рабочих и $m$ работ. И есть матрица умения: <<какой рабочий какие работы умеет делать>>. Нужно 
	максимально равномерно распределить работы между рабочими. То есть, каждой работе сопоставить рабочего, который 
	умеет делать эту работу, а кроме того минимизировать $\max\limits_{i=1\dots n} k_i$ где $k_i$ – количество 
	работ, выданных $i$-му рабочему.
\end{enumerate}
