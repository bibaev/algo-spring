\section*{Потоки и разрезы}
\begin{enumerate}
	\item Дан граф и выделенные вершины $s, t$. Нужно проверить, правда ли существует единственный минимальный s-t разрез.
	\begin{enumerate}
		\item $O(poly(V, E))$
		\item $O(E)$ при условии, что нам уже известен максимальный поток (с доказательством).
	\end{enumerate}
	
	\item В неориентированном графе без кратных рёбер необходимо удалить минимальное число рёбер так, чтобы 
	увеличилось количество компонент связности. $O(V \cdot Flow)$. Оцените время работы	того же алгоритма более 
	точно как $O(E^2)$.
	
	\item Есть ориентированный граф с начальной и конечной вершинами. В начальной вершине есть $K$ грузовиков. 
	Грузовикам нужно попасть в конечную вершину. Время дискретно. За единицу времени каждый грузовик или стоит на 
	месте, или перемещается в одну из соседних вершин. В любой вершине могут одновременно стоять несколько 
	грузовиков. По любому из рёбер в каждый момент времени должен ехать не более чем один грузовик. Минимизируйте 
	время, когда все грузовики окажутся в конечной вершине.
	\begin{enumerate}
		\item $O(poly(V, E, K))$
		\item $O(K(V + K)E)$
	\end{enumerate}
	
	\item Есть $n$ рабочих и $m$ работ. И есть матрица умения: <<какой рабочий какие работы умеет делать>>. Нужно 
	максимально равномерно распределить работы между рабочими. То есть, каждой работе сопоставить рабочего, который 
	умеет делать эту работу, а кроме того минимизировать $\max\limits_{i=1\dots n} k_i$ где $k_i$ – количество 
	работ, выданных $i$-му рабочему.
\end{enumerate}
